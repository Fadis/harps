\section{harps::input::Tracks$<$ OutputType, ReaderType, BufferType, track\_\-count $>$ Class Template Reference}
\label{classharps_1_1input_1_1Tracks}\index{harps::input::Tracks@{harps::input::Tracks}}
{\tt \#include $<$simple.hpp$>$}

\subsection*{Public Member Functions}
\begin{CompactItemize}
\item 
{\bf Tracks} (const char $\ast$\_\-sink)
\item 
void \textbf{setScore} (unsigned int \_\-target, const char $\ast$\_\-prog, const {\bf harps::input::Score} $\ast$\_\-score)\label{classharps_1_1input_1_1Tracks_194d2399ae3cb04e0c8cf231f03a2265}

\item 
void {\bf operator()} ()
\item 
bool {\bf isEnd} ()
\end{CompactItemize}
\subsection*{Classes}
\begin{CompactItemize}
\item 
class \textbf{TrackRunner}
\end{CompactItemize}


\subsection{Detailed Description}
\subsubsection*{template$<$typename OutputType, typename ReaderType, typename BufferType, unsigned int track\_\-count$>$ class harps::input::Tracks$<$ OutputType, ReaderType, BufferType, track\_\-count $>$}

複数のSimpleReaderを束ねて2つ以上の楽器を同時に演奏します。 

\subsection{Constructor \& Destructor Documentation}
\index{harps::input::Tracks@{harps::input::Tracks}!Tracks@{Tracks}}
\index{Tracks@{Tracks}!harps::input::Tracks@{harps::input::Tracks}}
\subsubsection[Tracks]{\setlength{\rightskip}{0pt plus 5cm}template$<$typename OutputType, typename ReaderType, typename BufferType, unsigned int track\_\-count$>$ {\bf harps::input::Tracks}$<$ OutputType, ReaderType, BufferType, track\_\-count $>$::{\bf Tracks} (const char $\ast$ {\em \_\-sink})\hspace{0.3cm}{\tt  [inline]}}\label{classharps_1_1input_1_1Tracks_9f4150bcd3a5ac116d4759909ea9a9cd}


コンストラクタ \begin{Desc}
\item[Parameters:]
\begin{description}
\item[{\em \_\-sink}]オーディオデータの出力先 \end{description}
\end{Desc}


\subsection{Member Function Documentation}
\index{harps::input::Tracks@{harps::input::Tracks}!operator()@{operator()}}
\index{operator()@{operator()}!harps::input::Tracks@{harps::input::Tracks}}
\subsubsection[operator()]{\setlength{\rightskip}{0pt plus 5cm}template$<$typename OutputType, typename ReaderType, typename BufferType, unsigned int track\_\-count$>$ void {\bf harps::input::Tracks}$<$ OutputType, ReaderType, BufferType, track\_\-count $>$::operator() ()\hspace{0.3cm}{\tt  [inline]}}\label{classharps_1_1input_1_1Tracks_39bfd26b91c4f04a0c6d9c7fdb01e1d9}


楽譜を読んでオーディオデータを生成し、指定された出力先に出力します。 

References harps::Buffer$<$ \_\-SampleType $>$::get(), harps::Normalizer$<$ tail $>$::getAmp(), harps::reduceNormalizingNoize(), and harps::Normalizer$<$ tail $>$::setValue().\index{harps::input::Tracks@{harps::input::Tracks}!isEnd@{isEnd}}
\index{isEnd@{isEnd}!harps::input::Tracks@{harps::input::Tracks}}
\subsubsection[isEnd]{\setlength{\rightskip}{0pt plus 5cm}template$<$typename OutputType, typename ReaderType, typename BufferType, unsigned int track\_\-count$>$ bool {\bf harps::input::Tracks}$<$ OutputType, ReaderType, BufferType, track\_\-count $>$::isEnd ()\hspace{0.3cm}{\tt  [inline]}}\label{classharps_1_1input_1_1Tracks_acb1ba476857eeaa4baaaa3e04c549de}


楽譜が終端に達しているかどうかを調べます \begin{Desc}
\item[Returns:]終端ならtrue そうでなければfalse \end{Desc}


The documentation for this class was generated from the following file:\begin{CompactItemize}
\item 
rc/harps/include/harps/input/simple.hpp\end{CompactItemize}
