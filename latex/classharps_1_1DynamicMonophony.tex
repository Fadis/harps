\section{harps::DynamicMonophony Class Reference}
\label{classharps_1_1DynamicMonophony}\index{harps::DynamicMonophony@{harps::DynamicMonophony}}
{\tt \#include $<$instrument.hpp$>$}

\subsection*{Public Member Functions}
\begin{CompactItemize}
\item 
{\bf DynamicMonophony} ()
\item 
virtual {\bf $\sim$DynamicMonophony} ()
\item 
{\footnotesize template$<$typename SampleType$>$ }\\void {\bf operator()} ({\bf Buffer}$<$ SampleType $>$ \&\_\-buffer)
\item 
int {\bf noteOn} (int \_\-note, float \_\-touch)
\item 
void {\bf noteOff} (int \_\-dummy=0)
\item 
void {\bf noteWait} (int \_\-dummy=0)
\item 
void {\bf setPitchBend} (float \_\-pitch)
\item 
void {\bf reset} ()
\item 
Note::NoteState {\bf getStatus} () const 
\item 
double {\bf getGlobalTime} ()
\item 
void {\bf programChange} (const char $\ast$\_\-filename)
\end{CompactItemize}


\subsection{Detailed Description}
モジュールの発音とクロックを管理するインスツルメントクラスです。 ダイナミックモノフォニーは外部トーンモジュールを読み込んで同時に1音を鳴らすことが出来ます。 

\subsection{Constructor \& Destructor Documentation}
\index{harps::DynamicMonophony@{harps::DynamicMonophony}!DynamicMonophony@{DynamicMonophony}}
\index{DynamicMonophony@{DynamicMonophony}!harps::DynamicMonophony@{harps::DynamicMonophony}}
\subsubsection[DynamicMonophony]{\setlength{\rightskip}{0pt plus 5cm}harps::DynamicMonophony::DynamicMonophony ()\hspace{0.3cm}{\tt  [inline]}}\label{classharps_1_1DynamicMonophony_1aa6282a44718f26647c48aa627a80f5}


コンストラクタ \index{harps::DynamicMonophony@{harps::DynamicMonophony}!$\sim$DynamicMonophony@{$\sim$DynamicMonophony}}
\index{$\sim$DynamicMonophony@{$\sim$DynamicMonophony}!harps::DynamicMonophony@{harps::DynamicMonophony}}
\subsubsection[$\sim$DynamicMonophony]{\setlength{\rightskip}{0pt plus 5cm}virtual harps::DynamicMonophony::$\sim$DynamicMonophony ()\hspace{0.3cm}{\tt  [inline, virtual]}}\label{classharps_1_1DynamicMonophony_c48c12ff80ee6f9088979fa13509e148}


デストラクタ 

\subsection{Member Function Documentation}
\index{harps::DynamicMonophony@{harps::DynamicMonophony}!operator()@{operator()}}
\index{operator()@{operator()}!harps::DynamicMonophony@{harps::DynamicMonophony}}
\subsubsection[operator()]{\setlength{\rightskip}{0pt plus 5cm}template$<$typename SampleType$>$ void harps::DynamicMonophony::operator() ({\bf Buffer}$<$ SampleType $>$ \& {\em \_\-buffer})\hspace{0.3cm}{\tt  [inline]}}\label{classharps_1_1DynamicMonophony_034e8a8eb8d8fafd8df36379fb1647dc}


モジュールを実行し、クロックを1ブロック分進めます。 \begin{Desc}
\item[Parameters:]
\begin{description}
\item[{\em \_\-buffer}]実行結果を出力するバッファブロック \end{description}
\end{Desc}
\index{harps::DynamicMonophony@{harps::DynamicMonophony}!noteOn@{noteOn}}
\index{noteOn@{noteOn}!harps::DynamicMonophony@{harps::DynamicMonophony}}
\subsubsection[noteOn]{\setlength{\rightskip}{0pt plus 5cm}int harps::DynamicMonophony::noteOn (int {\em \_\-note}, \/  float {\em \_\-touch})\hspace{0.3cm}{\tt  [inline]}}\label{classharps_1_1DynamicMonophony_183d62c092e5903c0e60b5ded5670290}


指定した鍵盤を押された状態にします。 \begin{Desc}
\item[Parameters:]
\begin{description}
\item[{\em \_\-note}]鍵盤(MIDIノートナンバーで指定) \item[{\em \_\-touch}]鍵盤を押す強さ \end{description}
\end{Desc}
\begin{Desc}
\item[Returns:]ノートID(モノフォニーは1音しか発音しないため、ノートIDは必ず0になります) \end{Desc}


References harps::Note::noteOn(), and harps::CurrentTime::noteOn().\index{harps::DynamicMonophony@{harps::DynamicMonophony}!noteOff@{noteOff}}
\index{noteOff@{noteOff}!harps::DynamicMonophony@{harps::DynamicMonophony}}
\subsubsection[noteOff]{\setlength{\rightskip}{0pt plus 5cm}void harps::DynamicMonophony::noteOff (int {\em \_\-dummy} = {\tt 0})\hspace{0.3cm}{\tt  [inline]}}\label{classharps_1_1DynamicMonophony_46144634cfb0eefe21660fdf5ac3baef}


現在押している鍵盤から指を離します。モジュールによっては鍵盤から指を離してもすぐに音は止まりません。音を完全に止めてインスツルメントを待機状態にするにはnoteWaitを呼び出す必要があります。 \begin{Desc}
\item[Parameters:]
\begin{description}
\item[{\em \_\-dummy}]ノートID(ポリフォニーとインターフェースを揃えるために用意されています。この値は無視されます) \end{description}
\end{Desc}


References harps::Note::noteOff(), and harps::CurrentTime::noteOff().\index{harps::DynamicMonophony@{harps::DynamicMonophony}!noteWait@{noteWait}}
\index{noteWait@{noteWait}!harps::DynamicMonophony@{harps::DynamicMonophony}}
\subsubsection[noteWait]{\setlength{\rightskip}{0pt plus 5cm}void harps::DynamicMonophony::noteWait (int {\em \_\-dummy} = {\tt 0})\hspace{0.3cm}{\tt  [inline]}}\label{classharps_1_1DynamicMonophony_25728aacf3836d4e1fe4e401b02021df}


発音処理を終了します。 \begin{Desc}
\item[Parameters:]
\begin{description}
\item[{\em \_\-dummy}]ノートID(ポリフォニーとインターフェースを揃えるために用意されています。この値は無視されます) \end{description}
\end{Desc}


References harps::Note::noteWait().\index{harps::DynamicMonophony@{harps::DynamicMonophony}!setPitchBend@{setPitchBend}}
\index{setPitchBend@{setPitchBend}!harps::DynamicMonophony@{harps::DynamicMonophony}}
\subsubsection[setPitchBend]{\setlength{\rightskip}{0pt plus 5cm}void harps::DynamicMonophony::setPitchBend (float {\em \_\-pitch})\hspace{0.3cm}{\tt  [inline]}}\label{classharps_1_1DynamicMonophony_4d304ca1400cb443cdb57a17cdedb3db}


ピッチベンドします。設定したピッチベンド値はresetされるまで引き継がれます。 \begin{Desc}
\item[Parameters:]
\begin{description}
\item[{\em \_\-pitch}]ピッチ(1.0で音階が1上がり、12.0で1オクターブ上がります) \end{description}
\end{Desc}


References harps::CurrentTime::setPitchBend().\index{harps::DynamicMonophony@{harps::DynamicMonophony}!reset@{reset}}
\index{reset@{reset}!harps::DynamicMonophony@{harps::DynamicMonophony}}
\subsubsection[reset]{\setlength{\rightskip}{0pt plus 5cm}void harps::DynamicMonophony::reset ()\hspace{0.3cm}{\tt  [inline]}}\label{classharps_1_1DynamicMonophony_9293fc3cfa3044994c65a06c0e5709dc}


インスツルメントをリセットします。設定されたピッチベンドは0に戻ります。 

References harps::CurrentTime::reset().\index{harps::DynamicMonophony@{harps::DynamicMonophony}!getStatus@{getStatus}}
\index{getStatus@{getStatus}!harps::DynamicMonophony@{harps::DynamicMonophony}}
\subsubsection[getStatus]{\setlength{\rightskip}{0pt plus 5cm}Note::NoteState harps::DynamicMonophony::getStatus () const\hspace{0.3cm}{\tt  [inline]}}\label{classharps_1_1DynamicMonophony_7691d2caaed5a624a33541a6c89f62a1}


現在のノートの状態を取得します。 \begin{Desc}
\item[Returns:]ノートの状態 \end{Desc}


References harps::Note::getStatus().\index{harps::DynamicMonophony@{harps::DynamicMonophony}!getGlobalTime@{getGlobalTime}}
\index{getGlobalTime@{getGlobalTime}!harps::DynamicMonophony@{harps::DynamicMonophony}}
\subsubsection[getGlobalTime]{\setlength{\rightskip}{0pt plus 5cm}double harps::DynamicMonophony::getGlobalTime ()\hspace{0.3cm}{\tt  [inline]}}\label{classharps_1_1DynamicMonophony_5891615a1418f7d7636469594c216f62}


現在の処理中の箇所の時刻を返します。 \begin{Desc}
\item[Returns:]現在時刻 \end{Desc}


References harps::CurrentTime::getTime().\index{harps::DynamicMonophony@{harps::DynamicMonophony}!programChange@{programChange}}
\index{programChange@{programChange}!harps::DynamicMonophony@{harps::DynamicMonophony}}
\subsubsection[programChange]{\setlength{\rightskip}{0pt plus 5cm}void harps::DynamicMonophony::programChange (const char $\ast$ {\em \_\-filename})\hspace{0.3cm}{\tt  [inline]}}\label{classharps_1_1DynamicMonophony_1906e8848e676602585df3f896166dca}


モジュールを変更します。 \begin{Desc}
\item[Parameters:]
\begin{description}
\item[{\em \_\-filename}]新しいモジュールのファイル名 \end{description}
\end{Desc}


The documentation for this class was generated from the following file:\begin{CompactItemize}
\item 
rc/harps/include/harps/instrument.hpp\end{CompactItemize}
